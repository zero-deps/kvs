\section{Ring Datastore}
\paragraph{}
Ring Application (also referred by the inner name RNG) is available as akka extension and makes your system part of the highly available, fault tolerant data distrubution cluster.

\paragraph{}
To configure RNG application on your cluster the next config options are available:

\begin{description}
  \item[quorum] Configured by array of three integer parameters N,W,R where
  \begin{description}
    \item[N] Number of nodes in bucket(in other words the number of copies).
    \item[R] Number of nodes that must  be participated in successful read operation.
    \item[W] Number of nodes for successful write.
  \end{description}
  To keep data consistent the quorums have to obey the following rules:
  \begin{enumerate}
    \item R + W > N
    \item W > N/2
  \end{enumerate}
  Or use the next hint:
  \begin{itemize}
    \item single node cluster [1,1,1]
    \item two nodes cluster [2,2,1]
    \item 3 and more nodes cluster [3,2,2]
  \end{itemize}
  if quorum fails on write operation, data will not be saved. So in case if 2 nodes and [2,2,1] after 1 node down the cluster becomes not writeable and readable.
  \item[buckets] Number of buckets for key. Think about this as about size of HashMap. In current implementation this value should not be changed after first setup.
  \item[virtual-nodes] Number of virtual nodes for each physical. In current implementation this value should not be changed after first setup.
  \item[hash-length] Lengths of hash from key. In current implementation this value should not be changed after first setup.
  \item[gather-timeout] Number of seconds that requested cluster will wait for response from another nodes.
  \item[ring-node-name] Role name that mark node as part of ring.
  \item[leveldb] Configuration of levelDB database used as backend for ring.
  \begin{description}
    \item[dir] directory location for levelDB storage.
    \item[fsync] if true levelDB will synchronise data to disk immediately.
  \end{description}
\end{description}

\paragraph{}
RNG provides default configuration for single node mode:

\begin{lstlisting}[language=json,caption=Example]
ring {
  quorum = [1, 1, 1]  //N,R,W
  buckets = 1024
  virtual-nodes = 128
  hash-length = 32
  gather-timeout = 3
  leveldb {
    dir = "rng_data_"${akka.remote.netty.tcp.hostname}"_"${akka.remote.netty.tcp.port}
    fsync = false
  }
}
\end{lstlisting}

\paragraph{}
In case default values are suitable for particular deployment, rewrite is not needed.
